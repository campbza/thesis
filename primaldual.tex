\documentclass[11pt]{article}
\usepackage[margin=1in]{geometry} 
\usepackage{amsmath,amsthm,amssymb}
\usepackage{mathtools}
 
\newcommand{\N}{\mathbb{N}}
\newcommand{\Z}{\mathbb{Z}}
 
\newenvironment{theorem}[2][Theorem]{\begin{trivlist}
\item[\hskip \labelsep {\bfseries #1}\hskip \labelsep {\bfseries #2.}]}{\end{trivlist}}
\newenvironment{lemma}[2][Lemma]{\begin{trivlist}
\item[\hskip \labelsep {\bfseries #1}\hskip \labelsep {\bfseries #2.}]}{\end{trivlist}}
\newenvironment{definition}[2][Definition]{\begin{trivlist}
\item[\hskip \labelsep {\bfseries #1}\hskip \labelsep {\bfseries #2.}]}{\end{trivlist}}
\newenvironment{problem}[2][Problem]{\begin{trivlist}
\item[\hskip \labelsep {\bfseries #1}\hskip \labelsep {\bfseries #2.}]}{\end{trivlist}}
\newenvironment{question}[2][Question]{\begin{trivlist}
\item[\hskip \labelsep {\bfseries #1}\hskip \labelsep {\bfseries #2.}]}{\end{trivlist}}
\newenvironment{corollary}[2][Corollary]{\begin{trivlist}
\item[\hskip \labelsep {\bfseries #1}\hskip \labelsep {\bfseries #2.}]}{\end{trivlist}}
\newenvironment{solution}[2][Solution]{\begin{trivlist}
\item[\hskip \labelsep {\bfseries #1}\hskip \labelsep {\bfseries #2.}]}{\end{trivlist}}

\begin{document}
\title{Duality Theory}
\author{Zachary Campbell}

\maketitle

Throughout this thesis we will use a powerful concept called 
\emph{linear-programming duality}. Duality gives us the tools we need to prove that a 
solution to a given linear program is optimal. An example of duality that may be familiar 
to the reader is the max-flow min-cut theorem. An instance of the maximum-flow problem can 
be described as a maximization linear program. We call the minimization linear program for 
minimum-cut the \emph{dual}. We know from the max-flow min-cut theorem that, given a graph 
$G$, the optimal solution to both of these problems is in fact the same. Duality theory will 
help to show us this connection. When discussing dual linear programs, we refer to the 
original as the ``primal.''

\begin{definition}{}
	Given a linear program in the form
	\begin{align}
		&\text{maximize} &\sum_{j=1}^{n} c_j x_j \\
		&\text{subject to}
		&\sum_{j=1}^{n} a_{ij} x_j &\leq b_i \text{ for } i = 1,2,...,m \\
		x_j &\geq 0 \text{ for } j = 1,2,...,n,
	\end{align}
	the dual is 
	\begin{align}
		\text{minimize}\; \; \sum_{i=1}^{m} b_i y_i \\
		\text{subject to}
		\sum_{i=1}^{m} a_{ij} y_i &\geq c_j \; \; \text{for } j = 1,2,...,n \\
		y_i &\geq 0 \; \; \text{for } i = 1,2,...,m.
	\end{align}
\end{definition}

It is natural to wonder how solutions to these two linear programs relate to each other. 
It turns out that their optima are always equal. Towards this conclusion, we have the 
following lemma, which tells us that any feasible solution to the primal linear program 
has value no greater than that of any feasible solution to the dual. In the following lemma,
assume that our primal LP is a maximization problem, and thus the dual is a minimization 
problem.

\begin{lemma}{(Weak-duality)}
	Let $\mathbf{x}$ be a primal-feasible solution, and let $\mathbf{y}$ be a dual-
	feasible solution. Then we have
	\[
		\sum_{j=1}^{n} c_j x_j \leq \sum_{i=1}^{m} b_i y_i.
	\]
\end{lemma}

\begin{proof}{}
	\begin{align*}
		\sum_{j=1}^{n} c_j x_j &\leq \sum_{j=1}^{n} \left( \sum_{i=1}^{m} a_{ij} y_i
		\right) x_j \\
		&= \sum_{i=1}^{m} \left( a_{ij} x_j \right) y_i \\
		&\leq \sum_{i=1}^{m} b_i y_i.
	\end{align*}
\end{proof}

This lemma gives rise to the idea of a ``gap'' between feasible primal and dual solutions. 
That is, one may ask, is there a gap between the largest primal solution and the smallest 
dual solution? This question is answered by the following theorem, which says that if there 
exists an optimal feasible solution to the primal, then there is no duality gap, which in 
turn implies (using the lemma) that the optima are in fact equal!

\begin{theorem}{(Strong duality)}
	If the primal has an optimal solution $\mathbf{x} = (x_1,...,x_n)$, then the dual 
	also has an optimal solution $\mathbf{y} = (y_1,...,y_m)$, and 
	\[
		\sum_i c_j x_j = \sum_i b_i y_i.
	\]
\end{theorem}



\end{document}
