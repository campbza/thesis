\documentclass[11pt]{article}
\usepackage[margin=1in]{geometry}
\usepackage{clrscode3e}
\usepackage{amsmath,amsthm,amssymb}
\usepackage{mathtools}
\usepackage{tikz}
\usepackage{float}
\usetikzlibrary{positioning,arrows}
\usepackage{setspace}
\usepackage{mathpazo}
\doublespacing
%\usepackage{tikz-berge}
%\usepackage{algorithm}
%\usepackage[noend]{algpseudocode}

 
\newcommand{\N}{\mathbb{N}}
\newcommand{\Z}{\mathbb{Z}}
\newcommand{\R}{\mathbb{R}}
\renewcommand{\'}{^{'}}
\renewcommand{\gets}{:=}
 
\newenvironment{theorem}[2][Theorem]{\begin{trivlist}
\item[\hskip \labelsep {\bfseries #1}\hskip \labelsep {\bfseries #2.}]}{\end{trivlist}}
\newenvironment{lemma}[2][Lemma]{\begin{trivlist}
\item[\hskip \labelsep {\bfseries #1}\hskip \labelsep {\bfseries #2.}]}{\end{trivlist}}
\newenvironment{definition}[2][Definition]{\begin{trivlist}
\item[\hskip \labelsep {\bfseries #1}\hskip \labelsep {\bfseries #2.}]}{\end{trivlist}}
\newenvironment{problem}[2][Problem]{\begin{trivlist}
\item[\hskip \labelsep {\bfseries #1}\hskip \labelsep {\bfseries #2.}]}{\end{trivlist}}
\newenvironment{question}[2][Question]{\begin{trivlist}
\item[\hskip \labelsep {\bfseries #1}\hskip \labelsep {\bfseries #2.}]}{\end{trivlist}}

\begin{document}
\title{Chapter 2}
\author{Zachary Campbell}

\maketitle
	In this chapter we will take a step back and look at primal-dual algorithms in more generality. 
	The goal will be to describe a method of solving a set of primal-dual algorithms for 
	\emph{network design problems}. Again, we will be restricting our attention to bipartite 
	graphs. In network design problems we are given a graph $G = (U,V,E)$ 
	and a cost $c_{uv}$ for each edge $(u,v)\in E$, and the goal is to find a minimum/maximum-cost 
	subset 
	$E\' \subset E$ that satisfies some criteria. Our maximum-cost matching problem is an example
	of this. There are other common examples that we will explore later on, but for now it suffices 
	to just think of these problems as choosing subsets of our graph according to some 
	stipulations. Throughout, we will be looking at undirected graphs.

\begin{section}{The Classical Primal-Dual Method}
	We begin by looking at what's known as the ``classical'' primal-dual method, which is concerned 
	with linear programs for polynomial-time solvable optimization problems. This will allow us 
	to build up a framework for a more general primal-dual method that we can use for approximation 
	algorithms - i.e. for problems that are known to be $NP$-hard. **(Not sure if this is within the 
	scope of the thesis -- discuss with Jim)\\
	Let's consider the linear program
	\begin{alignat}{2}
		& \text{minimize} & \mathbf{c}^{T}\mathbf{x} \\
		& \text{subject to } & A\mathbf{x} & \geq \mathbf{b} \\
		&& \mathbf{x} & \geq 0
	\end{alignat}
	and its dual
	\begin{alignat}{2}
		& \text{maximize} & \mathbf{b}^{T}\mathbf{y} \\
		& \text{subject to } & A^{T}\mathbf{y} & \leq \mathbf{c} \\
		&& \mathbf{y} & \geq 0.
	\end{alignat}
	We first define a concept that we will use throughout the rest of this thesis. 
	\begin{definition}{(Complementary slackness)}
		Given two linear programs in the form above, the \emph{primal complementary slackness 
		conditions} are the conditions which, given primal solution $\mathbf{x}$, 
		are necessary for a dual solution $\mathbf{y}$:
		\[
			x_j > 0 \implies A^{j}\mathbf{y} = c_j,
		\]
		where $A^{j}$ is the $j$th column of $A$. Similarly, the \emph{dual complementary 
		slackness conditions} are the conditions which, given dual solution $\mathbf{y}$, are 
		necessary for a primal solution $\mathbf{x}$:
		\[
			y_i > 0 \implies A_i\mathbf{x} = b_i,
		\]
		where $A_i$ is the $i$th row of $A$. Together, these conditions give us necessary and 
		sufficient conditions for solving the primal-dual system, which we will prove. The 
		(maximization) primal slackness variables are given by 
		$\mathbf{s} = \mathbf{b} - A\mathbf{x}$. The dual slackness variables are given by 
		$\mathbf{t} = A^{T}\mathbf{y} - \mathbf{c}$.
	\end{definition}
	\begin{theorem}{}
		[CITE THIS THEOREM]
		Let $\mathbf{x}$ be a primal feasible solution, and $\mathbf{y}$ a dual feasible 
		solution. Let $\mathbf{s}$ and $\mathbf{t}$ be the corresponding slackness variables. 
		Then $\mathbf{x}$ and $\mathbf{y}$ are optimal solutions if and only if the following 
		two conditions hold:
		\begin{align}
			x_jt_j &= 0 \quad \forall j \\
			y_is_i &= 0 \quad \forall i.
		\end{align}
	\end{theorem}
	\begin{proof}
		Let $u_i = y_is_i$ and $v_j = x_jt_j$, and $\mathbf{u} = \sum_i u_i$, 
		$\mathbf{v} = \sum_j v_j$. Then $\mathbf{u} = 0$ and $\mathbf{v} = 0$ if and only if 
		(7) and (8) hold. Also, 
		\begin{align*}
			\mathbf{u} + \mathbf{v} &= \sum y_is_i + \sum x_jt_j \\
						&= \sum y_i(b_i - A_ix_i) + \sum x_j (A^{T}_jy_j-c_j)\\
						&= \sum b_iy_i - \sum c_jx_j,
		\end{align*}
		so we get that $c^{T}\mathbf{x} = b^{T}\mathbf{y}$ if and only if $u + v = 0$, which 
		proves the statement.	
	\end{proof}
	The general ``tug-of-war'' between the primal and dual suggests an economic interpretation 
	of slackness conditions. We can think of our primal (maximization) problem as concerned with 
	profit given some constraints on resources, i.e. a resource allocation problem. The dual can 
	be interpreted as a valuation of the resources -- it tells us the availability of a resource, 
	and its price. So if we have optimal $\mathbf{x}$ and $\mathbf{y}$, we can interpret 
	slackness as follows: if there is slack in a constrained primal resource $i$ ($s_u > 0$), 
	then additional units of that resource must have no value ($y_u = 0$); if there is slack 
	in the dual price constraint ($t_v > 0$) there must be a shortage of that resource ($x_v = 0$).\\
	We now give an example of complementary slackness in action. Let's look back to our maximum 
	weight matching problem.Recall the primal linear program for maximum-weight matching:
	%Maximum matching ILP%
	\begin{alignat}{3}
		& \text{maximize } & \sum_{u,v} c_{uv} x_{uv}& \\
		& \text{subject to } \quad & \sum_{v} x_{uv} & \leq 1, & \quad \forall u\in U&, \\
				     &\quad & \sum_{u} x_{uv} & \leq 1, & \quad \forall v\in V &, \\
				&& x_{uv} & \geq 0.
	\end{alignat}
	and its dual
	%Vertex cover ILP%
	\begin{alignat}{3}
		& \text{minimize } & \sum_u y_u + \sum_v y_v& \\
		& \text{subject to } \quad & y_u + y_v & \geq c_{uv} & \quad \forall 
					u\in U,\ v\in V &, \\
				    && y_u,y_v & \geq 0.
	\end{alignat}
	The format here is a little different, since our primal is a maximization problem and the dual 
	is a minimization, but it's easy enough to reverse the roles. It's easy to see our 
	corresponding primal complementary slackness conditions are
	\begin{equation}
		x_{uv} > 0 \implies y_u + y_v = c_{uv}.
	\end{equation}
	The dual complementary slackness conditions are
	\begin{align}
		y_u > 0 &\implies \sum_v x_{uv} = 1,\\
		y_v > 0 &\implies \sum_u x_{uv} = 1.
	\end{align}
	In general, the slackness conditions guide us in our algorithm -- they tells us how, given a 
	solution to one of the problems, we should augment the solution to the other. For example, the 
	algorithm we presented for maximum-weight matching/minimum vertex cover intializes with 
	a solution to both the primal and dual that satisfies conditions (8) and (10); the algorithm 
	then at each step works to decrease the number of conditions in (9) that are unsatisfied, while 
	maintaining satisfiability of (8) and (10). This method is not unique to the Hungarian 
	algorithm. In fact, the Hungarian algorithm paved the way for this general method, which we 
	describe presently.\\
	Looking back at the original linear programs at the beginning of this chapter, suppose we have 
	a dual feasible solution $\mathbf{y}$. We can then state the problem of finding a feasible 
	primal solution $\mathbf{x}$ that obeys our complementary slackness conditions as another 
	\emph{restricted} linear program. Define the sets $A = \{v\ |\ A^{v}\mathbf{y} = c_v\}$ and 
	$B = \{u\ |\ y_u = 0\}$. So $A$ tells us which dual constraints (5) are tight, 
	given the solution $\mathbf{y}$, and $B$ tells us which $y_u$ are 0. What we want to do is 
	give a linear program to find a solution $\mathbf{x}$ that minimizes the 
	``violation'' of the complementary slackness conditions and the primal constraints. We will 
	have slack variables $s_u$ which will describe the difference between $A_u\mathbf{x}$ and $b_u$ 
	for $u\notin A$. We do this because we want to look at all $y_u > 0$ where the we do not 
	have that $A_u\mathbf{x} = b_u$. So part of our objective function will be to minimize the 
	sum of these $s_u$. We also want to minimize the sum over variables $x_j$ where $j\notin J$. 
	This is because we want to see if there are any $x_j$ such that $A^{j}\mathbf{y} \neq c_j$. 
	So we give the following restricted primal linear program:
	\begin{alignat}{3}
		& \text{minimize } & \sum_{i\notin I} s_i + \sum_{j\notin J} x_j & \\
		& \text{subject to } & A_i\mathbf{x} & \geq b_i & \quad i\in I &, \\
				     && A_i\mathbf{x} - b_i & = s_i & \quad i\notin I &, \\
				     && \mathbf{x} & \geq 0, \\
				     && \mathbf{s} & \geq 0.
	\end{alignat}
	Observe that if this restricted primal has a feasible solution $(\mathbf{x},\mathbf{s})$ such 
	that the objective function is 0, then $\mathbf{x}$ is a feasible primal solution that 
	satisfies the complementary slackness conditions for the dual solution $\mathbf{y}$. This 
	means that $\mathbf{x}$ and $\mathbf{y}$ are optimal primal and dual solutions. If, however, 
	the optimal solution to this restricted primal has value greater than 0, more work is required. 
	We can consider the dual of the restricted primal:
	\begin{alignat}{3}
		& \text{maximize } & \mathbf{b}^{T}\mathbf{w} & \\
		& \text{subject to } & A^{j}\mathbf{w} & \leq 0 & \quad j\in J &, \\
				     && A^{j}\mathbf{w} & \leq 1 & \quad j\notin J &, \\
				     && y_i\' & \geq -1 & \quad i\notin I &, \\
				     && y_i\' & \geq 0 & \quad i\in I &.
	\end{alignat}
	What we want here is to improve our dual solution. By assumption, the optimal solution to this 
	linear program's primal is greater than 0, so we know that this dual has a solution 
	$\mathbf{w}$ such that $\mathbf{b}^{T}\mathbf{w} > 0$. What we want is the existence of 
	some $\epsilon > 0$ such that $\mathbf{y}^{'} = \mathbf{y} + \epsilon \mathbf{w}$ is a 
	feasible dual solution. In particular, a solution of this form will be an improvement on our 
	original solution $\mathbf{y}$. We can calculate bounds on $\epsilon$ as follows. The two 
	conditions we must satisfy in order to maintain dual feasibility are that $y^{'} \geq 0$ and 
	$A^{T}y^{'} \leq c$. This means that we need 
	\begin{align}
		y_i + \epsilon w_i &\geq 0 \\
		A^{T}_j y + A^T_j \epsilon w & \leq c_j.
	\end{align}
	Let's consider the first one. When $w_i > 0$, we are fine; we need to be careful when 
	$w_i < 0$ since this could potentially violate the inequality. Solving in this way, we get 
	a first bound on $\epsilon$:
	\[
		\epsilon \leq \min_{i\in I: w_i < 0} (-y_i/w_i).
	\]
	Now let's address the second inequality. When $A^{T}_jw \leq 0$, we are defintely okay. We 
	need to be careful about violating the constraint when $A^{T}w > 0$. Thus, we can calculate a 
	second bound on $\epsilon$:
	\[
		\epsilon \leq \min_{j\in J: A^{T}_jw > 0} \frac{c_j-A^{T}_jy}{A^{T}_j w}.
	\]
	If we choose the lower of these two $\epsilon$ values, we obtain a new feasible dual solution 
	that has greater objective value. We can then work by reiterating the procedure, with the hope 
	that we find an optimal primal solution.\\
	It's not immediately clear why reducing our original linear programs to a series of linear 
	programs is heplful. However,  note that the vector $\mathbf{c}$ has totally disappeared in 
	the restricted primal and its dual. Recall that in the original linear program, $\mathbf{c}$ 
	gave us the edge-costs on our graph. So this method reworks our original weighted problem 
	into unweighted parts, which are easier to solve. Oftentimes, it is the case that 
	we can interpret these unweighted problems as purely combinatorial problems, which means that 
	instead of actually solving the problem with linear programming, we can solve it by 
	combinatorial 
	methods. Using a combinatorial algorithm to find a solution $\mathbf{x}$ that obeys the 
	complementary slackness conditions, or to find an improved dual solution $\mathbf{y}$, is 
	oftentimes more efficient.
\end{section}
	
\begin{section}{Primal-dual method for weighted matchings}
	Let us now look at an example of this method. We will look at a weighted matching problem, as 
	in the previous chapter, but this time we will look at \emph{minimizing} the cost of the 
	matching, instead of maximizing. We do this mainly because it illustrates something important 
	about the underlying structure of these matching problems. It will be easy to see how the 
	same method can be used for the case in which we want a maximimum matching. So the primal 
	linear program for a minimum weight perfect matching on a bipartite graph is given as follows. 
	\begin{alignat}{3}
		& \text{minimize } & \sum_{i,j} c_{ij} x_{ij}& \\
		& \text{subject to } \quad & \sum_{j} x_{ij} & \geq 1, & \quad \forall i\in L&, \\
				     &\quad & \sum_{i} x_{ij} & \geq 1, & \quad \forall j\in R &, \\
				&& x_{ij} & \in \{0,1\}.
	\end{alignat}
	Its dual is
	\begin{alignat}{3}
		& \text{maximize } & \sum_{i}u_i + \sum_jv_j& \\
		& \text{subject to } \quad & u_i + v_j & \leq c_{ij} & \quad \forall 
					i\in L,\ j\in R &, \\
				    && u_i,v_j & \in \{0,1\}.
	\end{alignat}
	We need to start with a dual feasible solution, and try to find a primal solution that 
	minimizes the violation of the constraints and slackness conditions. We can start with the 
	trivial dual solution of $u_i,v_j = 0$ for all $i,j$. Let's now think about our primal 
	complementary slackness. The set $J$ is given by $\{(i,j)\in E\ :\ u_i + v_j = c_{ij}\}$. 
	We know that these are the edges we want to include in our matching, and since we know our 
	linear program has integer solutions at extreme points of the polyhedron, let's specify that 
	$x_{ij} = 0$ for $(i,j)\notin J$. Now, our other slackness variables $s_i,s_j$ look like 
	\begin{align*}
		\sum_{j:(i,j)\in E} x_{ij} - s_i &= 1 \\
		\sum_{i:(i,j)\in E} x_{ij} - s_j &= 1.
	\end{align*}
	So we want to minimize over the sum of $s_i$ and $s_j$. Note that at this point, our set $I$ 
	consists of all vertices. So our restricted primal linear program is
	\begin{alignat}{3}
		& \text{minimize } & \sum_{i\in L} s_i + \sum_{j\in R} s_j & \\
		& \text{subject to } & \sum_j x_{ij} - s_i & = 1 & \quad \forall i &, \\
				     && \sum_i x_{ij} - s_j & = 1 & \quad \forall j &, \\
				     && x_{ij} & = 0 & \quad (i,j)\notin J, \\
				     && x_{ij} & \geq 0 & \quad (i,j)\in J, \\
				     && \mathbf{s} & \geq 0.
	\end{alignat}
	Let's first observe that all components of this restricted primal take on values 0 or 1, as 
	in the original primal. Moreover, note that we have turned a weighted problem into an 
	unweighted combinatorial problem. We've specified that we are not including any edge 
	$(i,j)\notin J$ in our matching, and we are trying to include as many $(i,j)\in J$ as possible 
	in our matching by minimizing the slackness variables. Note that the graph $G^{'} = (L,R,J)$ 
	is exactly the equality subgraph as defined in the previous chapter! In our Hungarian algorithm 
	we repeatedly sought to find maximum cardinality matchings within this subgraph, which is 
	exactly what this restricted primal is having us do. This tells us that the problems of maximum 
	weight matching and minimum weight matching only differ in the labeling we are specifying. The 
	underlying procedure for solving both of the problems is essentially the same. So if we find a 
	perfect matching in $G^{'}$, we will have found an $\mathbf{x}$ that obeys the complementary 
	slackness conditions, i.e. $\sum_i s_i + \sum_j s_j = 0$. Moreover, this implies that the 
	dual solution $\sum_i u_i + \sum_j v_j$ must be optimal as well.\\
	Now, if the solution $\sum_i s_i + \sum_j s_j > 0$, we do not have an optimal $\mathbf{x}$, 
	so we need to adjust our dual. We look at this now, in the dual linear program of the 
	restricted primal.
	\begin{alignat}{3}
		& \text{maximize } & \sum_{i\in L} w_i + \sum_{j\in R} w_j & \\
		& \text{subject to } & w_i + w_j & \leq 0 & \quad \forall (i,j)\in J &, \\
				     && w_i + w_j & \leq 1 & \quad \forall (i,j)\notin J &, \\
				     && w_i,w_j & \geq -1 & \quad i,j\notin I, \\
				     && w_i,w_j & \geq 0 & \quad i,j\in I, \\
				     && \mathbf{s} & \geq 0.
	\end{alignat}
	We now want to find an $\epsilon$ such that the solution $z = \sum_i u_i + \sum_j v_j + 
	\epsilon (\sum_i w_i + \sum_j w_j)$ is (1) feasible and (2) an improvement of the dual 
	objective. First of all, we know that since the restricted primal has solution $\geq 0$, 
	the solution to this dual will also be $\geq 0$. So we just need to worry about the condition 
	\[
		u_i + v_j + \epsilon (w_i + w_j) \leq c_{ij}.
	\]
	So we get that we at least need that $\epsilon \leq \min_{(i,j)\notin J: w_i + w_j > 0}
	\frac{c_{ij} - u_i - v_j}{w_i + w_j}$. We can refine this by noting that since 
	$0 < w_i + w_j \leq 1$ for $(i,j)\notin J$, we have $\epsilon = \min_{(i,j)\notin J} 
	(c_{ij} - u_i - v_j)$. Note that the negative of this is exactly the quantity we modify our 
	labeling by in the Hungarian algorithm in the previous chapter. Thus we've found an 
	$\epsilon$ that maintains dual feasibility, and increases the objectie function. We can use 
	this solution and revisit the restricted primal in order to look for an improved feasible 
	primal solution.
\end{section}
\end{document}
