\documentclass[11pt]{article}
\usepackage[margin=1in]{geometry}
\usepackage{clrscode3e}
\usepackage{amsmath,amsthm,amssymb}
\usepackage{mathtools}
\usepackage{tikz}
\usepackage{float}
\usetikzlibrary{positioning,arrows}
\usepackage{setspace}
\usepackage{mathpazo}
%\usepackage{tikz-berge}
%\usepackage{algorithm}
%\usepackage[noend]{algpseudocode}

 
\newcommand{\N}{\mathbb{N}}
\newcommand{\Z}{\mathbb{Z}}
\newcommand{\R}{\mathbb{R}}
\renewcommand{\'}{^{'}}
\renewcommand{\gets}{:=}
 
\newenvironment{theorem}[2][Theorem]{\begin{trivlist}
\item[\hskip \labelsep {\bfseries #1}\hskip \labelsep {\bfseries #2.}]}{\end{trivlist}}
\newenvironment{lemma}[2][Lemma]{\begin{trivlist}
\item[\hskip \labelsep {\bfseries #1}\hskip \labelsep {\bfseries #2.}]}{\end{trivlist}}
\newenvironment{definition}[2][Definition]{\begin{trivlist}
\item[\hskip \labelsep {\bfseries #1}\hskip \labelsep {\bfseries #2.}]}{\end{trivlist}}
\newenvironment{problem}[2][Problem]{\begin{trivlist}
\item[\hskip \labelsep {\bfseries #1}\hskip \labelsep {\bfseries #2.}]}{\end{trivlist}}
\newenvironment{question}[2][Question]{\begin{trivlist}
\item[\hskip \labelsep {\bfseries #1}\hskip \labelsep {\bfseries #2.}]}{\end{trivlist}}

\begin{document}
\title{Things that we need to do}
\author{Zachary Campbell}

\maketitle

\begin{section}{$\epsilon$ calculation in primal dual method}
	Given the two linear programs 
	\begin{alignat}{2}
		& \text{minimize} & \mathbf{c}^{T}\mathbf{x} \\
		& \text{subject to } & A\mathbf{x} & \geq \mathbf{b} \\
		&& \mathbf{x} & \geq 0
	\end{alignat}
	\begin{alignat}{2}
		& \text{maximize} & \mathbf{b}^{T}\mathbf{y} \\
		& \text{subject to } & A^{T}\mathbf{y} & \leq \mathbf{c} \\
		&& \mathbf{y} & \geq 0.
	\end{alignat}
	and the corresponding restricted primal and dual
	\begin{alignat}{3}
		& \text{minimize } & \sum_{i\notin I} s_i + \sum_{j\notin J} x_j & \\
		& \text{subject to } & A_i\mathbf{x} & \geq b_i & \quad i\in I &, \\
				     && A_i\mathbf{x} - b_i & = s_i & \quad i\notin I &, \\
				     && \mathbf{x} & \geq 0, \\
				     && \mathbf{s} & \geq 0.
	\end{alignat}
	\begin{alignat}{3}
		& \text{maximize } & \mathbf{b}^{T}\mathbf{y}\' & \\
		& \text{subject to } & A^{j}\mathbf{y}\' & \leq 0 & \quad j\in J &, \\
				     && A^{j}\mathbf{y}\' & \leq 1 & \quad j\notin J &, \\
				     && y_i\' & \geq -1 & \quad i\notin I &, \\
				     && y_i\' & \geq 0 & \quad i\in I &.
	\end{alignat}
	Recall that we define the sets $I = \{i | y_i = 0\}$ and $J = \{j | A^{T}_j y = c_j\}$.
	Goemans and Williamson claim that the dual solution $\mathbf{y}^{''} = \mathbf{y} + \epsilon 
	\mathbf{y}^{'}$ is an improved, dual feasible solution when $\epsilon$ is given as follows. 
	\begin{itemize}
		\item Their first claim: by definition of $I$, $\mathbf{y}^{''} \geq 0$ if $\epsilon 
			\leq 
			\min_{i\notin I: y_i^{'} < 0} (-y_i/y_i^{'})$. I'm not quite sure how they 
			make this conclusion. Note that $y^{''}\geq 0$ is 
			$y_i + \epsilon y_i^{'} \geq 0$. Now, there are two cases, either $y_i^{'} 
			\geq 0$, in which case we have $\epsilon \geq (-y_j / y_j^{'})$, which is 
			negative. In the other case, we have that $y_i^{'} < 0$, so we have 
			$y_i - \epsilon y_i^{'} \geq 0$, which gives us $\epsilon \leq (-y_j/y_j^{'})$.
			Why do we minimize $\epsilon$ over the latter, and not the former?
		\item Their second claim: by definition of $J$, $A^{T}y^{''} \leq c$ if 
			$\epsilon \leq \min_{j\notin J: A^{T}_jy^{'} > 0} (c_j - A^{T}_j y)/
			(A^{T}_jy^{'})$. It's again a straightforward calculation to get the quantity
			$(c_j - A^{T}_j y)/(A^{T}_jy^{'})$, but I do not understand why they are 
			only taking the minimum over $j\notin J: A^{T}_jy^{'} > 0$.
	\end{itemize}
	I need help understanding these details in this calculation in order to perform a similar 
	calculation for our modified restricted programs.
\end{section}
\begin{section}{Our altered primal-dual problem}
	In our case for maximum weight matchings, our primal problem is a maximization problem, and 
	the dual is a minimization problem. My candidate restricted programs for this are given 
	as follows
	\begin{alignat}{3}
		& \text{minimize } & \sum_{i\in L} s_i + \sum_{j\in R} s_j & \\
		& \text{subject to } & \sum_j x_{ij} - s_i & = 1 & \quad \forall i &, \\
				     && \sum_i x_{ij} - s_j & = 1 & \quad \forall j &, \\
				     && x_{ij} & = 0 & \quad (i,j)\notin J, \\
				     && x_{ij} & \geq 0 & \quad (i,j)\in J, \\
				     && s & \geq 0.
	\end{alignat}
	\begin{alignat}{3}
		& \text{maximize } & \sum_{i\in L} u_{i}^{'} + \sum_{j\in R} v_{j}^{'} & \\
		& \text{subject to } & u_i^{'} + v_j^{'} & \leq 0 & \quad (i,j)\in J &, \\
				     & u_i^{'} + v_j^{'} & \leq 1 & \quad (i,j)\notin J &, \\
				     && u_i^{'},v_j^{'} &\geq -1 & \quad i,j\notin I &, \\ 
				     && u_{i}^{'},v_j^{'} &\geq 0 & \quad i,j\in I &.
	\end{alignat}
	These are not a mirror of what Goemans and Williamson give, but I think they reduce to the 
	same combinatorial problem: find a maximum cardinality matching in the subgraph $G = (L,R,J)$ 
	(ie the equality subgraph). What we need to show (I think) is that, first of all, this dual 
	solution is greater than zero, and, furthermore, we can find an $\epsilon$ such that 
	$\sum_i u_i + \sum_j v_j \geq \sum_i u_i + \sum_j v_j + \epsilon (\sum_i u_i^{'} + \sum_j 
	v_j^{'})$, where the RHS of the inequality is a dual feasible solution. Showing this shows that 
	our system improves the dual solution when we can't find an optimal primal. I think the 
	$\epsilon$ will be something like $\min_{(i,j)\in E\setminus J} (u_i + v_j - w_{ij})$ (this 
	would correspond to what happens in the Hungarian).
\end{section}
\begin{section}{How is the auction algorithm primal-dual??}
	Original authors of the algorithm claim that it is a modified Hungarian algorithm, and authors 
	of the blog post claim it is primal-dual. I see this in a naive way, as we are dealing with 
	adjusting labelings and overdemanded sets as given in Hall's Marriage Theorem, but I CAN NOT 
	figure out how this algorithm relates to the primal-dual linear programs.
\end{section}
\begin{section}{Other less immediate things}
	\begin{itemize}
		\item Brief section on flows, Ford-Fulkerson.
		\item More info on the economics of matchings -- more in depth analysis of auctions, 
			how the algorithm given is a good model of actual multi-item auctions.
		\item Possible expansion on how we can use the primal-dual method for approximating 
			NP-hard combinatorial problems. Don't need to go into detail here, but the 
			thesis would not feel complete without at least dedicating a couple of pages 
			to this.
	\end{itemize}
\end{section}
\end{document}

