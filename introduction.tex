This thesis explores the curious mathematics behind a theorem covered in a typical course in 
algorithms and data structures, namely the max-flow min-cut theorem. The aim of the thesis is to 
demonstrate the intimate relationship between such problems in combinatorial optimization and linear 
programming.

In order to do this, we first 
review concepts in elementary graph theory (bipartite graphs and matchings) and introduce the topic 
of linear programming. We then look at a problem called the vertex cover problem, which we will discover 
is fundamentally related to the matching problem on bipartite graphs. It turns out that given a solution 
to one of these problems, we can construct a solution to the other. We discover this relationship 
using duality theory, an important concept in linear programming. 

As a consequence of the duality 
theorems, we show that the max-flow min-cut relationship in flow networks is simply a special case of 
linear programming duality. After this, we turn our attention to the main algorithm in this thesis:
the Hungarian algorithm. Its invention was a significant development in combinatorial optimization 
in that its generalization led to several other techniques based on linear programming and linear 
programming duality. 

Finally, we take a step back in our final chapter and look at the general 
``primal-dual method'' inspired by the Hungarian algorithm. This method uses general linear programming 
principles to design algorithms for network design problems (essentially problems on weighted graphs in 
which one desires to choose some optimal subset of the graph). The main result here is that, quite 
magically, we are able to turn these \emph{weighted} problems into equivalent unweighted problems, which 
are often much easier to solve. We finish by describing a class of auction algorithms which use this 
primal-dual methodology.

This thesis began as an endeavour to understand the Hungarian algorithm. From this, we learned 
quite a lot about its influence on the world of combinatorial optimization and linear programming. 
Although we restrict our attention to polynomial-time solvable problems in this thesis, the tools 
developed generally extend to, and are geared towards, approximation algorithms for $NP$-hard 
problems. 
