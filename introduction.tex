This thesis explores the curious mathematics behind a theorem covered in a typical course in 
algorithms and data structures, namely the Max-flow/Min-cut theorem. The aim of the thesis is to 
demonstrate the intimate relationship between problems in combinatorial optimization, and a field 
called linear programming. \\
In order to do this, we develop useful tools along the way. We first 
review concepts in elementary graph theory (bipartite graphs and matchings), and introduce the topic 
of linear programming. We then look at a problem called the vertex problem, which we will discover 
is fundamentally related to the matching problem on bipartite graphs. We discover this relationship 
using duality theory, an important aspect of linear programming. As a consequence of our duality 
theorems we are show that the max-flow/min-cut relationship in flow networks is just a special case of 
linear programming duality. After this, we turn our attention to the main algorithm in this thesis:
the Hungarian algorithm. This algorithm for solving weighted bipartite matchings was developed in the 
50s by Harold Kuhn. This algorithm is significant in that it used duality theory as its fundamental 
design principle; in many ways it predicted the use of linear programs to solve combinatorial 
optimization problems. Finally, we take a step back in our final chapter and look at the general 
``primal-dual method'' inspired by the Hungarian algorithm. This method uses general linear programming 
principles to design algorithms for network design problems. The main result here is that, quite 
magically, we are able to turn \emph{weighted} problems into equivalent unweighted versions, which 
are often easier to solve. We finish by describing a class of auction algorithms which use this 
primal-dual methodology.\\
This thesis began as an endeavour to understand the Hungarian algorithm. From this, we learned 
quite a lot about it's influence on the world of combinatorial optimization and linear programming. 
Although we restrict our attention to polynomial-time solvable problems in this thesis, the tools 
developed generally extend to, and are geared towards, approximation algorithms for $NP$-hard 
problems. 
