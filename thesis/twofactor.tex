\documentclass[11pt]{article}
\usepackage[margin=1in]{geometry} 
\usepackage{amsmath,amsthm,amssymb}
\usepackage{mathtools}
\usepackage{algorithmic, algorithm}
 
\newcommand{\N}{\mathbb{N}}
\newcommand{\Z}{\mathbb{Z}}
 
\newenvironment{theorem}[2][Theorem]{\begin{trivlist}
\item[\hskip \labelsep {\bfseries #1}\hskip \labelsep {\bfseries #2.}]}{\end{trivlist}}
\newenvironment{lemma}[2][Lemma]{\begin{trivlist}
\item[\hskip \labelsep {\bfseries #1}\hskip \labelsep {\bfseries #2.}]}{\end{trivlist}}
\newenvironment{exercise}[2][Exercise]{\begin{trivlist}
\item[\hskip \labelsep {\bfseries #1}\hskip \labelsep {\bfseries #2.}]}{\end{trivlist}}
\newenvironment{problem}[2][Problem]{\begin{trivlist}
\item[\hskip \labelsep {\bfseries #1}\hskip \labelsep {\bfseries #2.}]}{\end{trivlist}}
\newenvironment{question}[2][Question]{\begin{trivlist}
\item[\hskip \labelsep {\bfseries #1}\hskip \labelsep {\bfseries #2.}]}{\end{trivlist}}
\newenvironment{corollary}[2][Corollary]{\begin{trivlist}
\item[\hskip \labelsep {\bfseries #1}\hskip \labelsep {\bfseries #2.}]}{\end{trivlist}}
\newenvironment{solution}[2][Solution]{\begin{trivlist}
\item[\hskip \labelsep {\bfseries #1}\hskip \labelsep {\bfseries #2.}]}{\end{trivlist}}
\newenvironment{definition}[2][Definition]{\begin{trivlist}
\item[\hskip \labelsep {\bfseries #1}\hskip \labelsep {\bfseries #2.}]}{\end{trivlist}}



\begin{document}
\title{Finding a 2-factor}
\author{Zachary Campbell}

\maketitle

\begin{definition}{(2-factor)}
	A \emph{2-factor} of a graph $G = (V,E)$ is a spanning subgraph of $G$ for which all 
	vertices have degree two.
\end{definition}

We will give a linear program for finding 2-factors in bipartite graphs.\\

Let $G = (V,E)$ be our graph, $|V| = n$, and $|E| = m$. Our objective is the following: 

\begin{align*}
	\text{maximize } &\sum_{(u,v)\in E} x_{(u,v)} \\
	\text{subject to } &\sum_{v\in E(u)} x_{(u,v)} \leq 2 \\
			   &0\leq x_{(u,v)} \leq 1.
\end{align*}

We really want integer solutions to this program; unfortunately, integer linear programs 
are difficult to solve in general. However, in our specific case we can show that our 
program attains integer solutions, using the following theorem:

\begin{theorem}{}
	If $M$ is a totally unimodular matrix and $b$ is an integral vector, then for 
	each vector $c$ the linear programming problem
	\[
		\max \{c^{T}\ |\ Mx\leq b \}
	\]
	has an integral optimum solution (assuming the solution space is bounded).
\end{theorem}



\end{document}
