\documentclass[11pt]{article}
\usepackage[margin=1in]{geometry} 
\usepackage{amsmath,amsthm,amssymb}
\usepackage{mathtools}
\usepackage{algorithmic, algorithm}
 
\newcommand{\N}{\mathbb{N}}
\newcommand{\Z}{\mathbb{Z}}
 
\newenvironment{theorem}[2][Theorem]{\begin{trivlist}
\item[\hskip \labelsep {\bfseries #1}\hskip \labelsep {\bfseries #2.}]}{\end{trivlist}}
\newenvironment{lemma}[2][Lemma]{\begin{trivlist}
\item[\hskip \labelsep {\bfseries #1}\hskip \labelsep {\bfseries #2.}]}{\end{trivlist}}
\newenvironment{exercise}[2][Exercise]{\begin{trivlist}
\item[\hskip \labelsep {\bfseries #1}\hskip \labelsep {\bfseries #2.}]}{\end{trivlist}}
\newenvironment{problem}[2][Problem]{\begin{trivlist}
\item[\hskip \labelsep {\bfseries #1}\hskip \labelsep {\bfseries #2.}]}{\end{trivlist}}
\newenvironment{question}[2][Question]{\begin{trivlist}
\item[\hskip \labelsep {\bfseries #1}\hskip \labelsep {\bfseries #2.}]}{\end{trivlist}}
\newenvironment{corollary}[2][Corollary]{\begin{trivlist}
\item[\hskip \labelsep {\bfseries #1}\hskip \labelsep {\bfseries #2.}]}{\end{trivlist}}
\newenvironment{solution}[2][Solution]{\begin{trivlist}
\item[\hskip \labelsep {\bfseries #1}\hskip \labelsep {\bfseries #2.}]}{\end{trivlist}}
\newenvironment{definition}[2][Definition]{\begin{trivlist}
\item[\hskip \labelsep {\bfseries #1}\hskip \labelsep {\bfseries #2.}]}{\end{trivlist}}



\begin{document}
\title{Maximum flow}
\author{Zachary Campbell}

\maketitle

Let $G = (V,E),s,t)$ be a graph $G = (V,E)$ with source $s$ and sink $t$. We will formulate the 
max-flow problem as a linear program.\\
Let $c:E\to \mathbb{R}^{+}$ be our cost function. We will make a slight modification by 
adding an edge from $t$ to $s$ that has infinite capacity. The objective is to maximize the flow 
along this edge. We have a variable $f_{uv}$ for every edge 
$(u,v)\in E$. This is the \emph{flow} over $(u,v)$. The problem then becomes:

\begin{align*}
	\text{maximize } &f_{ts}\\
	\text{subject to } &f_{ij}\leq c_{ij},          &(i,j)\in E\\
	&\sum_{j:(j,i)\in E} f_{ji} - \sum_{j:(i,j)\in E} f_{ij} \leq 0,     &i\in V\\
	&f_{ij}\geq 0,           &(i,j)\in E.
\end{align*}

The second condition says that the total flow into a node $i$ is at most the total flow out of $i$. 
When this inequality holds for each node, it must in fact be satisfied with equality at each 
node, because a deficit in flow balance at one node would imply a surplus at some other node, so flow 
is conserved.\\
We will now look at the dual of this linear program. Let variables $d_{ij}$ and $p_i$ correspond to 
the two types of inequalities in the primal. View these variables as distance labels on arcs and 
potentials on nodes, resp. Then the dual is:

\begin{align*}
	\text{minimize } &\sum_{(i,j)\in E} c_{ij}d_{ij} \\
	\text{subject to } &d_{ij} - p_i + p_j \geq 0,     &(i,j)\in E \\
	&p_s - p_t \geq 1, \\
	&d_{ij}\geq 0,   &(i,j)\in E \\
	&p_i\geq 0,    &i\in V. 
\end{align*}

Let's restrict this to an ILP that seeks 0/1 variable solutions. 
Then the only way to satisfy the inequality $p_s - p_t\geq 1$ is to set $p_s = 1$ and $p_t = 0$.




\end{document}
