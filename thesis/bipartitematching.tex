\documentclass[11pt]{article}
\usepackage[margin=1in]{geometry} 
\usepackage{amsmath,amsthm,amssymb}
\usepackage{mathtools}
\usepackage{algorithmic, algorithm}
 
\newcommand{\N}{\mathbb{N}}
\newcommand{\Z}{\mathbb{Z}}
 
\newenvironment{theorem}[2][Theorem]{\begin{trivlist}
\item[\hskip \labelsep {\bfseries #1}\hskip \labelsep {\bfseries #2.}]}{\end{trivlist}}
\newenvironment{lemma}[2][Lemma]{\begin{trivlist}
\item[\hskip \labelsep {\bfseries #1}\hskip \labelsep {\bfseries #2.}]}{\end{trivlist}}
\newenvironment{exercise}[2][Exercise]{\begin{trivlist}
\item[\hskip \labelsep {\bfseries #1}\hskip \labelsep {\bfseries #2.}]}{\end{trivlist}}
\newenvironment{problem}[2][Problem]{\begin{trivlist}
\item[\hskip \labelsep {\bfseries #1}\hskip \labelsep {\bfseries #2.}]}{\end{trivlist}}
\newenvironment{question}[2][Question]{\begin{trivlist}
\item[\hskip \labelsep {\bfseries #1}\hskip \labelsep {\bfseries #2.}]}{\end{trivlist}}
\newenvironment{corollary}[2][Corollary]{\begin{trivlist}
\item[\hskip \labelsep {\bfseries #1}\hskip \labelsep {\bfseries #2.}]}{\end{trivlist}}
\newenvironment{solution}[2][Solution]{\begin{trivlist}
\item[\hskip \labelsep {\bfseries #1}\hskip \labelsep {\bfseries #2.}]}{\end{trivlist}}
\newenvironment{definition}[2][Definition]{\begin{trivlist}
\item[\hskip \labelsep {\bfseries #1}\hskip \labelsep {\bfseries #2.}]}{\end{trivlist}}



\begin{document}
\title{Bipartite Matching}
\author{Zachary Campbell}

\maketitle

\begin{definition}{(Bipartite graph)}
	A graph $G = (V, E)$ is \emph{bipartite} if the set of vertices $V$ can be partitioned into 
	two disjoint and independent sets $A$ and $B$ such that no edge in $E$ has both endpoints in 
	the same set of the partition.
\end{definition}

Note that square-grid graphs are bipartite, but triangle-grid graphs are not. 

\begin{definition}{(Matching)}
	A matching $M\subset E$ is a collection of edges such that every vertex of $V$ is incident 
	to at most one edge of $M$, i.e. a set of edges in $G$ that do not share vertices.
\end{definition}

\begin{section}{Maximum cardinality matching problem}
	In this problem, $G$ is unweighted and our goal is to find a matching $M$ of maximum size.
	As a start, we want to prove optimality of a matching. To this end, we could find upper 
	bounds on the size of any matching and hope that the smallest of these upper bounds  is 
	equal to the size of the largest matching. This is a duality concept that will prove 
	useful. \\
	A \emph{vertex cover} is a set $C$ of vertices such that all edges $e$ of $E$ are incident to 
	at least on vertex of $C$. In other words, there is no edge completely contained in 
	$V\setminus C$. Any matching $M$ is at most the size of any vertex cover, since any 
	vertex cover $C$ must contain at least one of the endpoints of each edge in $M$. This 
	shows \emph{weak duality}. This leads to a theorem:

	\begin{theorem}{}
		For any bipartite graph $G$, the maximum size of a matching is equal to the minimum size 
		of a vertex cover on $G$.
	\end{theorem}

	\begin{definition}{(Alternating path)}
		An alternating path with respect to a matching $M$ is a path that alternates between 
		edges in $M$ and edges in $E\setminus M$.
	\end{definition}

	\begin{definition}{(Augmenting path)}
		An augmenting path with respect to a matching $M$ is an alternation path in which the 
		first and last vertices are exposed.
	\end{definition}

	A useful property of augmenting paths is the following: let $P$ be an augmenting path with 
	respefct to a matching $M$ (so the edges in $P$ alternate between $M$ and $G\setminus M$), 
	and set $M^{'} = (M\setminus P)\cup (P\setminus M)$, we get a matching $M^{'}$ with 
	$|M^{'}| = |M| + 1$. We say that we have \emph{augmented M along P}. This leads to a theorem:

	\begin{theorem}{}
		A matching $M$ is maximum if and only if there are no augmenting paths with 
		respect to $M$.
	\end{theorem}

	\begin{proof}{}
		Let $P$ be an augmenting path w.r.t. a matching $M$. Let $M^{'} = (M\setminus P)\cup 
		(P\setminus M)$. Then $M^{'}$ is a matching with size greater than $M$. This 
		contradicts the maximality of $M$.

		Now suppose $M$ is not maximum. Let $M^{'}$ be a maximum matching ($|M^{'}| > |M|$). 
		Let $Q = (M\setminus M^{'})\cup (M^{'}\setminus M)$. Then:
		\begin{itemize}
			\item $Q$ has more edges from $M^{'}$ than from $M$, since $|M^{'}| > |M|$ 
				means that $|M^{'} \setminus M| > |M \setminus M^{'}|$.
			\item Each vertex is incident to at most one edge in $M\cap Q$ and one edge 
				in $M^{'}\cap Q$.
			\item So $Q$ is composed of cycles and paths that alternate between edges from 
				$M$ and $M$. This path is an augmenting path w.r.t. $M$.
		\end{itemize}
		Hence there must exist an augmenting path $P$ w.r.t $M$, which is a contradiction.
	\end{proof}

	This motivates the following algorithm:
	\begin{algorithm}
		\caption{Maximum unweighted matching}
	\begin{algorithmic}
		\STATE Input: $G = (V, E)$
		\STATE $M = \emptyset$
		\WHILE{there exists an augmenting path $P$ w.r.t. $M$}
		    \STATE{augment $M$ along $P$ to get a matching $M^{'}$}
		    \STATE{set $M = M^{'}$}
		\ENDWHILE
		\RETURN $M$
	\end{algorithmic}
	\end{algorithm}

        \begin{subsection}{On the existence of augmenting paths}
		We still need to address the existence, and our ability to find, augmenting paths.
		We do this as follows: for a matching $M$ in $G$, direct an edge $A\to B$ if it 
		does ot belong to $M$, otherwise direct it $B\to A$. Call this new directed graph 
		$D$.

		\begin{theorem}{}
			There exists an augmenting path in $G$ w.r.t. a matching $M$ iff there 
			exists a directed path in $D$ between an exposed vertex in $A$ and an 
			exposed vertex in $B$.
		\end{theorem}

		\begin{proof}{}
			Suppose there exists an augmenting path $P$ w.r.t. $M$. This means that 
			$M^{'} = (M\setminus P)\cup (P\setminus M)$ is a new matching.
		

\end{section}


\end{document}
